
%!TEX root = ../thesis.tex
% ******************************* Thesis Appendix B ********************************

\chapter{Experimental Methods}

\section{Experimental methods for Chapter 4}
Experimental methods compiled by Mariya Chhatriwala, as in \cite{cuomo2020single}.

\subsection{Cell culture for maintenance and differentiation}

Human iPSC lines were thawed for differentiation and maintained in Essential 8 (E8) media (LifeTech) on vitronectin (StemCell Technologies, \#07180) coated Corning plates according to the manufacturer's instructions.  
Cells were passaged at least twice after thawing and always 3 - 4 days before plating for differentiation to ensure all the cell lines in each experiment were growing at a similar rate prior to differentiation. 
Gelatine/MEF coated plates were prepared 24 – 48 hours before plating for differentiation by incubating plates with 0.1\% gelatine for 20 minutes at room temp. 
The gelatine was then aspirated and plates were incubated in MEF medium overnight at 37°C.  
Immediately prior to plating cells, plates were washed once with D-PBS to remove any residual MEF medium.  
To plate for endoderm differentiation, cells were washed once with D-PBS and dissociated using StemPro Accutase (Life Technologies, A1110501) at 37°C for 3 - 5 min. 
Colonies were fully dissociated through gentle pipetting. 
Cells were resuspended in MEF medium, passed through a 40 µm cell strainer, and pelleted gently by centrifuging at 300 x g for 5 min. Cells were re-suspended in E8 media and plated at a density of 15,000 cells per cm2  on gelatin/MEF coated plates \cite{hannan2013production, yiangou2018human} in the presence of 10 µM Rock inhibitor – Y27632 (Sigma, \#Y0503 - 5 mg). 
Media was replaced with fresh E8 free of Rock inhibitor every 24 hours post plating. 
Differentiation into definitive endoderm commenced 72 hours post plating.  
Cells were washed 1x gently with D-PBS to remove residual E8.  
Cells were then incubated in CDM-PVA containing 100 ng/mL ActivinA (made in house), 80 ng/mL FGF2 (made in house), 10 ng/mL BMP4 (R\&D systems, \#314-BP-050), 10 µm Ly294002 (Promega, \#V1201), and 3 µM CHIR99201 (Selleckchem, \#S1263) for 24 hours (day 1).  
After 24 hours, the day 1 media was replaced with CDM-PVA containing 100 ng/mL ActivinA, 80 ng/mL FGF2, 10 ng/mL BMP4, and 10 µm Ly294002 for another 24 hours (day 2).  
Day 2 media was then replaced with RPMI/B27 containing 100 ng/mL ActivinA and 80 ng/mL FGF2 for another 24 hours (day 3) \cite{hannan2013production}. 
The overall efficiency of the differentiation protocol was validated using reference lines with good and poor differentiation capacity, respectively.  
All media was filtered through 0.22 µm filters prior to use.

\subsection{Single cell preparation and sorting for scRNAseq}
Cells were dissociated into single cells using Accutase and washed once with MEF medium as described above when plating cells for differentiation. 
For all subsequent steps, cells were kept on ice to avoid degradation.  Approximately 1 x 106 cells were re-suspended in PBS + 2\% BSA + 2 mM EDTA (FACS buffer); BSA and PBS were nuclease-free. 
For staining of cell surface markers, 1 x 106 cells were re-suspended in 100 µL of ice-cold FACS buffer containing 20 µL anti-Tra-1-60 antibody (BD Pharmingen, BD560380)  and 5 µL of anti-CXCR4 antibody (eBioscience 12-9999-42), and were placed on ice for 30 min. 
Cells were protected from light during staining and all subsequent steps. 
Cells were washed with 5 mL of FACS buffer, passed through a 35 µm filter to remove clumps, and re-suspended in 300 µL of FACS buffer for live cell sorting on the BD Influx Cell Sorter (BD Biosciences).
Live/dead marker 7AAD (eBioscience 00-6993) was added immediately prior to analysis at a concentration of 2 µL/mL and only living cells were considered when determining differentiation capacities. 
Living cells stained with 7AAD but not TRA-1-60 or CXCR4 were used as gating controls. Data for TRA-1-60 and CXCR4 staining were available for 31,724 cells, of the total 36,044. 
Single-cell transcriptomes of sorted cells were assayed as follows: reverse transcription and cDNA amplification was performed according to the SmartSeq2 protocol \cite{picelli2013smart}, and library preparation was performed using an Illumina Nextera kit. 
Samples were sequenced using paired-end 75bp reads on an Illumina HiSeq 2500 machine (one lane of sequencing per 384 well plate).

\subsection{ChIP-seq experiments and data processing}
\label{sec:endodiff_chipseq}

ChIP-seq was performed using FUCCI-Human Embryonic Stem Cells (FUCCI-hESCs, H9 from WiCell) in a modified endoderm differentiation protocol to that used for the iPSC differentiations (see details below). 
Cells were grown in defined culture conditions as described previously \cite{brons2007derivation}. 
Pluripotent cells were maintained in Chemically Defined Media with BSA (CDM-BSA) supplemented with 10ng/ml recombinant Activin A and 12ng/ml recombinant FGF2 (both from Dr. Marko Hyvonen, Dept. of Biochemistry, University of Cambridge) on 0.1\% Gelatin and MEF media coated plates. 
Cells were passaged every 4-6 days with collagenase IV as clumps of 50-100 cells. 
The culture media was replaced 48 hours after the split and then every 24 hours. \\

The generation of FUCCI-hESC lines has been described in \cite{pauklin2013cell} and are based on the FUCCI system described in \cite{sakaue2008visualizing}. 
hESCs were differentiated into endoderm as previously described \cite{vallier2009early}. 
Following FACS sorting, Early G1 (EG1) cells were collected and immediately placed into the endoderm differentiation media and time-points were collected every 24h up to 72h.
Endoderm specification was performed in CDM with Polyvynilic acid (CDM-PVA) supplemented with 20ng/ml FGF2, 10$\mu$M Ly-294002 (Promega), 100ng/ml Activin A, and 10ng/ml BMP4 (R\&D). \\

We performed ChIP as described previously \cite{pauklin2016initiation}. 
For ChIP-sequencing, ChIP for various histone marks (H3K4me3, H3K27me3, H3K4me1, H3K27ac, H3K36me3) (see \textbf{Table \ref{tab:endodiff_chipseq_antibodies}} for antibodies) was performed on two biological replicates per condition. 
At the end of the ChIP protocol, fragments between 100bp and 400bp were used to prepare barcoded sequencing libraries. 
10ng of input material for each condition were also used for library preparation and later used as a control during peak calls. 
The libraries were generated by performing 8 PCR cycles for all samples. 
Equimolar amounts of each library were pooled and this multiplexed library was diluted to 8pM before sequencing using an Illumina HiSeq 2000 with 75bp paired-end reads. \\

Reads were mapped to GRCh38 reference assembly using BWA \cite{li2009fast}. 
Only reads with mapping quality score $\geq$ 10 and aligned to autosomal and sex chromosomes were kept for further processing. 
Peak calling analysis \cite{bailey2013practical} was performed using PeakRanger \cite{feng2011peakranger}, and only the peaks that were reproducible at an FDR of $\leq$ 0.05 in two biological replicates were selected for further processing. 
Peak calling was done using appropriate controls with the tool peakranger 1.18 in modes ranger (H3K4me3, H3K27ac; `-l 316 -b 200 -q 0.05'), ccat (H3K27me3; ‘-l 316 --win\_size 1000 --win\_step 100 --min\_count 70 --min\_score 7 -q 0.05’) and bcp (H3K4me1, H3K36me3; ‘-l 316’). 
Adjacent peak regions closer than 40 bp were merged using the BEDTools suite \cite{quinlan2010bedtools}, and those overlapping ENCODE blacklisted regions were filtered out (ENCODE Excludable Mappability Regions \cite{encode2012integrated}). 
Finally, peaks were converted to GRCh37 coordinates using UCSC LiftOver. 

\section{Experimental methods for Chapter 5}
Experimental methods written by Julie Jerber, as in \cite{jerber2020population}.

\subsection{Human iPSC culture}
Feeder-free human iPSCs were obtained from the HipSci project \cite{kilpinen2017common}. 
Lines were thawed onto tissue culture-treated plates (Corning, 3516) coated with 10 µg/mL VitronectinXF (StemCell Technologies, 07180) using complete Essential 8 (E8) medium (Thermo Fisher, A1517001) and 10 µM Rock inhibitor (Sigma, Y0503). 
Cells were expanded in E8 medium for 2 passages using 0.5 µM EDTA pH 8.0 (Thermo Fisher, 15575-020) for cell dissociation. 

\subsection{Pooling and differentiation of midbrain dopaminergic neurons}
iPSC colonies were dissociated into a single-cell suspension using Accutase (Thermo Fisher, A11105-01) and resuspended in E8 medium containing 10 µM Rock inhibitor. 
Cells were counted using an automated cell counter (Chemometec NC-200) and a cell suspension containing an equal amount of each iPSC line was prepared in E8 medium containing 10 µM Rock inhibitor and seeded at 2 x 105 cells per cm2 on 0.01\% Geltrex- (Thermo Fisher, A1413202) coated plates. 
Each pool of lines contained between 7 to 24 donors. 
24h after plating, neuronal differentiation of the pooled lines to a midbrain lineage was performed as described by12 with minor modifications: 1. SHH C25II was replaced by 100nM SAG (Tocris, 6390) in the neuronal induction phase. 
2. On day 20, the cells were passaged with Accutase containing 20 units/mL of papain (Worthington, LK00031765) and plated at 3.5 x 105 cells per cm2 on 0.01\% Geltrex-coated plates for final maturation.

\subsection{Rotenone stimulation}
On day 51 of differentiation, the cells were exposed for 24h to freshly prepared 0.1 μM rotenone (Sigma, R8875, purity HPLC $\geq$ 95\%) diluted in neuronal maturation medium \cite{kriks2011dopamine}. 
The final DMSO concentration was 0.01\% in all exposure conditions. 
Unstimulated control samples (i.e. DMSO only) were taken concurrently.

\subsection{Generation of cerebral organoids}
Cerebral organoids were generated according to the enCOR method as previously described by \cite{lancaster2017guided}. 
Briefly, one pool of 18 iPSC lines was thawed and expanded for 1 passage before seeding 18,000 cells onto PLGA microfilaments prepared from Vicryl sutures. 
STEMdiff Cerebral Organoid kit (Stem Cell Technologies, 08570) was used for organoid culture with timing according to manufacturer's suggestion and Matrigel embedding as previously described57. 
From day 35 onward the medium was supplemented with 2\% dissolved Matrigel basement membrane (Corning, 354234), and processed for scRNA-seq after 113 days of culture. 

\subsection{Generation of single cell suspensions for sequencing}
On harvesting days, the cells were washed once with 1X DPBS (Thermo Fisher, 14190-144) before adding either Accutase (day 11) or Accutase containing 20 units/mL of papain (days 30 and 52). 
The cells were incubated at 37°C for up to 20 min (day 11) or up to 35 min (days 30 and 52) before adding DMEM:F12 (Thermo Fisher Scientific, 10565-018) supplemented with 10 µM Rock inhibitor and 33 μg/mL DNase I (Worthington, LK003170, only for days 30 and 52). 
The cells were dissociated using a P1000 and collected in a 15 mL tube capped with a 40 µm cell strainer. 
After centrifugation, the cells were resuspended in 1X DPBS containing 0.04\% BSA (Sigma, A0281) and washed 3 additional times in 1X DPBS containing 0.04\% BSA. Single-cell suspensions were counted using an automated cell counter (Chemometec NC-200) and concentrations adjusted to 5 x 105 cells/mL. \\

Organoids were washed twice in 1X DPBS before adding EBSS (Worthington, LK003188)  dissociation buffer containing 19 U/mL of papain, 50 μg/mL of DNase I and 22.5X of Accutase.
Organoids were incubated in a shaking block (750 rpm) at 37°C for 30 min. 
Every 10 min, the organoids were triturated using a P1000 and BSA-coated  pipette tips until large clumps were dissociated. 
Dissociated organoids were transferred into a new tube capped with a 40 µm cell strainer and pelleted for 4 min at 300g. 
After centrifugation, the cells were resuspended in EBSS containing 50μg/mL of DNase I and 2 mg/mL ovomucoid (Worthington, LK003150). 
0.5 volume of EBSS, followed by 0.5 volume of 20 mg/mL ovomucoid were added to the top of the cell suspension and the cells were mixed by flicking the tube. 
After centrifugation, the cells were resuspended in 1X DPBS containing 0.04\% BSA. 
Single-cell suspensions were counted using an automated cell counter and concentrations adjusted to 5 x105 cells/mL. 

\subsection{Immunohistochemistry}
Cells were fixed in 4\% paraformaldehyde (Thermo Fisher Scientific, 28908) for 15 min, rinsed 3 times with PBS1X (Sigma, D8662) and blocked with 5\% normal donkey serum (NDS; AbD Serotec, C06SBZ) in PBST (PBS1X + 0.1\% Triton X-100, Sigma, 93420) for 2h at room temperature. Primary antibodies were diluted in PBST containing 1\% NDS and incubated overnight at 4°C.
Cells were washed 5 times with PBS1X and incubated with secondary antibodies diluted in PBS1X for 45 min at room temperature.  
Cells were washed 3 more times with PBS1X and Hoechst (Thermo Fisher Scientific, H3569) was used to visualize cell nuclei. Image acquisition was performed using Cellomics array scan VTI (Thermo Fisher Scientific).
The following antibodies were used: 
FOXA2 (Santa Cruz, sc101060 - 1/100)
LMX1A (Millipore, AB10533 - 1/500)
TH (Santa Cruz, sc-25269 - 1/200)
MAP2 (Abcam, 5392 - 1/2000)
Donkey anti-chicken AF647 (Thermo Fisher Scientific, A21449)
Donkey anti-mouse AF488 (Thermo Fisher Scientific, A11008)
Donkey anti-mouse AF555 (Thermo Fisher Scientific, A31570)
Donkey anti-rabbit AF488 (Thermo Fisher Scientific, A21206)
Donkey anti-rabbit AF555 (Thermo Fisher Scientific, A27039)

\subsection{Chromium 10x Genomics library and sequencing}
Single cell suspensions were processed by the Chromium Controller (10x Genomics) using Chromium Single Cell 3’ Reagent Kit v2 (PN-120237). 
On average, 15,000 cells from each 10x reaction were directly loaded into one inlet of the 10x Genomics chip. 
All the steps were performed according to the manufacturer's specifications. 
Barcoded libraries were sequenced using HiSeq4000 (Illumina, one lane per 10x chip position) with 50bp or 75bp paired end reads to an average depth of 40,000-60,000 reads per cell.