% ************************** Thesis Abstract *****************************
\begin{abstract}

Over the last fifteen years, genome-wide association studies (GWAS) have been used to identify thousands of DNA variants associated with complex traits and diseases, by exploiting naturally occurring genetic variation in large populations of individuals. 
More recently, similar approaches have been applied to \gls{rnaseq} data to find variants associated with expression level, called expression quantitative trait loci (eQTL).
Recent advances in experimental techniques have provided an unprecedented opportunity to measure gene expression at the single cell level, and the chance to study cellular heterogeneity. 
This represents a remarkable advance over traditional bulk sequencing methods, particularly to study cell fate commitment events in development.
The challenge of studying early human development is partially overcome by advances in stem cell technologies.
In particular, induced pluripotent stem cells (iPSCs) and cells derived therefrom represent a fantastic system to study development \textit{in vitro}.
In this thesis, I investigate the computational challenges of using single cell expression profiles to perform \gls{eqtl} mapping, and provide suitable approaches for the identification of cell type and context-specific eQTL using single cell expression profiles.
I further explore the application of such methods across a range of human iPSC-derived cell types, using data from the \gls{hipsci} project. 

\end{abstract}
