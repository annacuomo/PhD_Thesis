%!TEX root = ../thesis.tex
%*******************************************************************************
%****************************** Seventh Chapter *********************************
%*******************************************************************************

\chapter{Discussion}  %Title of the Seventh Chapter

\section{Bridging the gap}

Still a large gap in our understanding of function from genotype to phenotype.
Translation and clinical application still far away
Single cell data helps
from healthy to diseased

\section{Differentiation efficiency of iPSC differentiation protocols}

Room for improvement in our understanding on how and when the protocols work

\section{Key differences and commonalities between the two datasets}

This section aims to highlight common aspects and differences between the two datasets described in chapters 4 and 5 respectively, and illustrate how they complement each other for our purposes.

\subsection{iPSC lines used}
In both cases, we used human iPSC lines from the HipSci consortium (see 1.4). 
In fact, 45 lines were included in both studies.
Through the HipSci resource, we also have access to bulk RNA sequencing for most of the same lines.\\

% add table for numbers of lines in common?

All HipSci lines are originated from fibroblasts, and we use all lines that were converted to a feeder-free medium (ref).
Lines are from both sexes, and mostly healthy. 
We do use a small portion of diseased lines in the endoderm differentiation protocol: 12 lines (out of 126) had monongenic diabetes, which did not affect any of the analyses. 

\subsection{Experimental design}
The pooling approach from the first study was replicated in the second, at a larger scale: in the endoderm protocol, 4 to 6 lines were differentiated together; in the midbrain neuronal differentiation protocol 9 to 21 lines were.
In both cases, a subset of lines were differentiated in more than one pool (18/126, 35/215 respectively).\\

An obvious difference is the length of the differentiation protocols.
The definitive endoderm protocol is extremely short, lasting only 72h.
As a consequence, considering different differentiation rates for different cells, we essentially have a continuos mapping of cells from pluripotent (0h i.e. day0) to definitive endoderm (72h i.e. day3).
The midbrain dopaminergic neuronal differentiation protocol, on the other hand lasts nearly two months: 52 days.
At the end, cells are much more differentiated and better resemble \textit{in vivo }mature cell types.
We collected cells at day 11, day 30, and at the last time point, day 52.
This time points are quite separate and discrete.


\subsection{scRNA-seq technology used}

Moreover, the scRNA-seq technology used in the two studies is different.\\

In the endoderm differentiation study, for which data generation started in 2015, cells were sequenced using a plate-based scRNA-seq technology (SmartSeq2).\\

The neuronal differentiation study is more recent (generation started in 2018) and thus used a technology that is more popular now (mainly because of UMI addition and scalability, see 1.3 for more detail), droplet-based (10X) sequencing.

\subsection{Scale of study}
A direct consequence of the technology used is a change in scale.
SmartSeq2 is limited to 384 cell per plate
Using 10X, one run can take up to 4,000 cells.
We can observe this difference very clearly in our two datasets, made up by little less than 40,000 cells and just over a million (!) cells, respectively.\\

We did also assess many more donors in the neuronal differentiation data: 215 (vs 125 from the endoderm study).

\section{Discussion}

The fact that iPSC differentiation protocols...