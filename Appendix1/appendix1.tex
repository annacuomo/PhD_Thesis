%!TEX root = ../thesis.tex
% ******************************* Thesis Appendix A ****************************
\chapter{Supplementary Tables} 

% \section{Additional results Chapter 3}

% results on smaller gene selection?

\section{Additional results Chapter 4}

\begin{table}[h]
    \centering
    \begin{tabular}{c | c c c c c}
     & eGenes & genes tested & cells & unique donors & unique samples*  \\
    \hline
    iPSC (day0) & 1,833 & 10,840 & 9,661 & 111 & 136 \\
    mesendo & 1,702 & 10,924 & 9,809 & 123 & 224 \\
    defendo & 1,342 & 10,901 & 10,924 & 116 & 238\\
    transitioning & 227 & 10,924 & 6,387 & 118 & 313 \\
    day1 & 1,181 & 10,787 & 9,443 & 111 & 138 \\
    day2 & 718 & 10,788 & 8,455 & 105 & 116 \\
    day3 & 631 & 10,765 & 8,485 & 108 & 127 \\
    \end{tabular}
    \caption[Summary of the type and number of eQTL]{\textbf{Summary of the type and number of eQTL}.\\
    Including all eQTL discovered based on single cell RNA traits, both at the level of our computationally-defined stages (iPSC, mesendo and defendo) and the time points of collection by design (day1, day2 and day3).
    We note that in the stage definition a number of cells ($\sim$20\%) were excluded and considered to not belong to any well defined stage (see \textbf{page \pageref{fig:endodiff_stages}}), exclusively for the purposes of stage-level eQTL mapping.
    We included here an eQTL map of such transitioning population, which as expected was a very underpowered analysis.
    Additionally, we added results for day2 cells, for completeness.
    Shown are the number of genes that were considered for QTL mapping, as well as the number of genes for which a QTL was detected.
    *Samples are donor-experiment(-time point) combinations which were effectively used for testing (see \textbf{section \ref{sec:endodiff_eqtl}}).}
    \label{tab:endodiff_eqtl_summary}
\end{table}
% number of eqtl, sample size etc?


\begin{table}[h]
    \centering
    \begin{tabular}{c c c}
     Antibody raised against & Catalog number & Company  \\
    \hline
    Histone H3 & ab1791 & Abcam \\
    Histone H3 (tri methyl K4) & ab8580 & Abcam \\
    Histone H3 (tri methyl K27) & C15200181 (MAb-181-050)
 & Diagenode\\
    Histone H3 (mono methyl K4) & ab8895 & Abcam \\
    Histone H3 (acetyl K27) & ab4729 & Abcam \\
    Histone H3 (tri methyl K36) & ab9050 & Abcam \\
    \end{tabular}
    \caption[Antibodies used for ChIP-seq experiments]{\textbf{Antibodies used for ChIP-seq experiments}.\\
    .}
    \label{tab:endodiff_chipseq_antibodies}
\end{table}

\clearpage

\section{Additional information for Chapter 5}

\begin{table}[h]
    \centering
    \begin{tabular}{c c c c c}
    trait & category & study & n & n (replication)  \\
    \hline
    Alzheimer's Disease (late onset) &  neurodegenerative & \cite{lambert2013meta}  & 55,134 & 19,884\\
    Parkinson's Disease  &  neurodegenerative & \cite{nalls2019expanding} & 442,271 & -\\
    Schizophrenia   &  neurodevelopmental & \cite{schizophrenia2014biological} & 150,064 & -\\
    Bipolar disorder & neurodevelopmental & \cite{hou2016genome} & 34,950 & 5,305\\
    Neuroticism & personality & \cite{okbay2016genetic} & 170,911 & -\\
    Neuroticism & personality & \cite{luciano2018association} & 329,821 & 122,867\\
    Neuroticism & personality & \cite{turley2018multi} & 168,105 & -\\
    Depression (broad) & behavioural & \cite{howard2018genome} & 322,580 & -\\
    Educational attainment & intelligence & \cite{davies2016genome} & 111,114 & - \\
    Paternal history of Alzheimer's disease &  neurodegenerative & \cite{marioni2018gwas} & 260,279 & - \\
    Family history of Alzheimer's disease &  neurodegenerative & \cite{marioni2018gwas} & 314,278 & -\\
    Maternal history of Alzheimer's disease &  neurodegenerative & \cite{marioni2018gwas} & 288,676 & -\\
    Depressed affect & behavioural & \cite{nagel2018meta} & 357,957 & -\\
    Cognitive performance & intelligence & \cite{lee2018gene} & 257,841 & -\\
    Sleeplessness / insomnia & personality & \cite{neale2018gwas} & 360,738 & -\\
    Nervous feelings & behavioural & \cite{neale2018gwas} & 351,829 & -\\
    Worrier / anxious feelings & behavioural & \cite{neale2018gwas} & 351,833 & - \\
    Tense / 'highly strung' & behavioural & \cite{neale2018gwas} & 350,159 & -\\
    Suffer from 'nerves' & behavioural & \cite{neale2018gwas} & 348,082 & -\\
    Neuroticism score & personality & \cite{neale2018gwas} & 293,006 & - \\
    Intelligence questions\footnotemark & intelligence & \cite{neale2018gwas} & 117,131 & -\\
    Risk taking & behavioural & \cite{neale2018gwas} & 348,549 & -\\
    College or University degree & intelligence & \cite{neale2018gwas} & 357,549 & - \\
    A levels/AS levels or equivalent & intelligence & \cite{neale2018gwas} & 357,549 & -\\
    Other professional qualifications  & intelligence & \cite{neale2018gwas} & 357,549 & -\\
    \end{tabular}
    \caption[Neurological traits used for the colocalisation analysis]{\textbf{Neurological traits used for the colocalisation analysis}.\\
    Table compiled by Natsuhiko Kumasaka.
    Traits used for the colocalisation analysis in \textbf{Chapter 5} (\textbf{section \ref{sec:neuroseq_coloc}}).
    The corresponding publication and sample size (both for the initial study and, when available for a replication study) are indicated.
    Additionally, traits are divided into broad categories.}
    \label{tab:coloc_neuro_traits}
\end{table}

% \newpage

\footnotetext{Number of fluid intelligence questions attempted within time limit.}

% \section{Additional results Chapter 5}



% \subsection*{Detailed QC steps}

% \subsection*{scRNA-seq quality control and processing}

% Adaptors of raw scRNA-seq reads were trimmed using Trim Galore! (refs), using default settings. Trimmed reads were mapped to the human reference genome build 37 using STAR (ref) (version: 020201) in two-pass alignment mode, using the default settings proposed by the ENCODE consortium (STAR manual). Gene-level expression quantification was performed using Salmon (ref) (version: 0.8.2), using the “--seqBias”, “--gcBias” and “VBOpt” options using ENSEMBL transcripts (built 75)(ref)

% Gene-level expression quantification was performed using Salmon (ref) (version: 0.8.2), using the “--seqBias”, “--gcBias” and “VBOpt” options using ENSEMBL transcripts (built 75)(ref). Transcript-level expression values were summarised at a gene level (estimated counts per million (CPM)) and quality control of scRNA-seq data was performed with the scater Bioconductor package in R (ref). Cells were retained for downstream analyses if they had at least 50,000 counts from endogenous genes, at least 5,000 genes with non-zero expression, less than 90pct of counts came from the 100 highest-expressed genes, less than 15pct of reads mapping to mitochondrial (MT) genes, they had a Salmon mapping rate of at least 60pct, and if the cell was successfully assigned to a donor. Dead cells as identified based on 7AAD staining were discarded. Size factor normalisation of counts was performed using the scran Bioconductor package in R (ref). Expressed genes with an HGNC symbol were retained for analysis, where expressed genes in each batch of samples were defined based on (i) raw count >100 in at least one cell prior to QC and (ii) average log2(CPM+1) >1 after QC. Normalised CPM data were log transformed (log2(CPM+1)) for all downstream analyses. The joint dataset was investigated for outlying cell lines or experimental batches, which identified no clear groups of outlying cells (Supplementary Fig. 21, 22). 


% % \begin{enumerate}
% % \item	Download the TeXLive ISO (2.2GB) from\\
% % \href{https://www.tug.org/texlive/}{https://www.tug.org/texlive/}
% % \item	Download WinCDEmu (if you don't have a virtual drive) from \\
% % \href{http://wincdemu.sysprogs.org/download/}
% % {http://wincdemu.sysprogs.org/download/}
% % \item	To install Windows CD Emulator follow the instructions at\\
% % \href{http://wincdemu.sysprogs.org/tutorials/install/}
% % {http://wincdemu.sysprogs.org/tutorials/install/}
% % \item	Right click the iso and mount it using the WinCDEmu as shown in \\
% % \href{http://wincdemu.sysprogs.org/tutorials/mount/}{
% % http://wincdemu.sysprogs.org/tutorials/mount/}
% % \item	Open your virtual drive and run setup.pl
% % \end{enumerate}

% % or

% % \subsection*{Basic MikTeX - \TeX~ distribution}
% % \begin{enumerate}
% % \item	Download Basic-MiK\TeX (32bit or 64bit) from\\
% % \href{http://miktex.org/download}{http://miktex.org/download}
% % \item	Run the installer 
% % \item	To add a new package go to Start >> All Programs >> MikTex >> Maintenance (Admin) and choose Package Manager
% % \item	Select or search for packages to install
% % \end{enumerate}

% % \subsection*{TexStudio - \TeX~ editor}
% % \begin{enumerate}
% % \item	Download TexStudio from\\
% % \href{http://texstudio.sourceforge.net/\#downloads}
% % {http://texstudio.sourceforge.net/\#downloads} 
% % \item	Run the installer
% % \end{enumerate}

% % \section*{Mac OS X}
% % \subsection*{MacTeX - \TeX~ distribution}
% % \begin{enumerate}
% % \item	Download the file from\\
% % \href{https://www.tug.org/mactex/}{https://www.tug.org/mactex/}
% % \item	Extract and double click to run the installer. It does the entire configuration, sit back and relax.
% % \end{enumerate}

% % \subsection*{TexStudio - \TeX~ editor}
% % \begin{enumerate}
% % \item	Download TexStudio from\\
% % \href{http://texstudio.sourceforge.net/\#downloads}
% % {http://texstudio.sourceforge.net/\#downloads} 
% % \item	Extract and Start
% % \end{enumerate}


% % \section*{Unix/Linux}
% % \subsection*{TeXLive - \TeX~ distribution}
% % \subsubsection*{Getting the distribution:}
% % \begin{enumerate}
% % \item	TexLive can be downloaded from\\
% % \href{http://www.tug.org/texlive/acquire-netinstall.html}
% % {http://www.tug.org/texlive/acquire-netinstall.html}.
% % \item	TexLive is provided by most operating system you can use (rpm,apt-get or yum) to get TexLive distributions
% % \end{enumerate}

% % \subsubsection*{Installation}
% % \begin{enumerate}
% % \item	Mount the ISO file in the mnt directory
% % \begin{verbatim}
% % mount -t iso9660 -o ro,loop,noauto /your/texlive####.iso /mnt
% % \end{verbatim}

% % \item	Install wget on your OS (use rpm, apt-get or yum install)
% % \item	Run the installer script install-tl.
% % \begin{verbatim}
% % 	cd /your/download/directory
% % 	./install-tl
% % \end{verbatim}
% % \item	Enter command `i' for installation

% % \item	Post-Installation configuration:\\
% % \href{http://www.tug.org/texlive/doc/texlive-en/texlive-en.html\#x1-320003.4.1}
% % {http://www.tug.org/texlive/doc/texlive-en/texlive-en.html\#x1-320003.4.1} 
% % \item	Set the path for the directory of TexLive binaries in your .bashrc file
% % \end{enumerate}

% % \subsubsection*{For 32bit OS}
% % For Bourne-compatible shells such as bash, and using Intel x86 GNU/Linux and a default directory setup as an example, the file to edit might be \begin{verbatim}
% % edit $~/.bashrc file and add following lines
% % PATH=/usr/local/texlive/2011/bin/i386-linux:$PATH; 
% % export PATH 
% % MANPATH=/usr/local/texlive/2011/texmf/doc/man:$MANPATH;
% % export MANPATH 
% % INFOPATH=/usr/local/texlive/2011/texmf/doc/info:$INFOPATH;
% % export INFOPATH
% % \end{verbatim}
% % \subsubsection*{For 64bit OS}
% % \begin{verbatim}
% % edit $~/.bashrc file and add following lines
% % PATH=/usr/local/texlive/2011/bin/x86_64-linux:$PATH;
% % export PATH 
% % MANPATH=/usr/local/texlive/2011/texmf/doc/man:$MANPATH;
% % export MANPATH 
% % INFOPATH=/usr/local/texlive/2011/texmf/doc/info:$INFOPATH;
% % export INFOPATH

% % \end{verbatim}



% % %\subsection{Installing directly using Linux packages} 
% % \subsubsection*{Fedora/RedHat/CentOS:}
% % \begin{verbatim} 
% % sudo yum install texlive 
% % sudo yum install psutils 
% % \end{verbatim}


% % \subsubsection*{SUSE:}
% % \begin{verbatim}
% % sudo zypper install texlive
% % \end{verbatim}


% % \subsubsection*{Debian/Ubuntu:}
% % \begin{verbatim} 
% % sudo apt-get install texlive texlive-latex-extra 
% % sudo apt-get install psutils
% % \end{verbatim}
