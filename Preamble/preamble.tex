% ******************************************************************************
% ****************************** Custom Margin *********************************

% Add `custommargin' in the document class options to use this section
% Set {innerside margin / outerside margin / topmargin / bottom margin}  and
% other page dimensions
\ifsetCustomMargin
  \RequirePackage[left=37mm,right=30mm,top=35mm,bottom=30mm]{geometry}
  \setFancyHdr % To apply fancy header after geometry package is loaded
\fi

% Add spaces between paragraphs
%\setlength{\parskip}{0.5em}
% Ragged bottom avoids extra whitespaces between paragraphs
\raggedbottom
% To remove the excess top spacing for enumeration, list and description
%\usepackage{enumitem}
%\setlist[enumerate,itemize,description]{topsep=0em}

% No indent before paragraph
\setlength\parindent{0pt}

% *****************************************************************************
% ******************* Fonts (like different typewriter fonts etc.)*************

% Add `customfont' in the document class option to use this section

\ifsetCustomFont
  % Set your custom font here and use `customfont' in options. Leave empty to
  % load computer modern font (default LaTeX font).
  %\RequirePackage{helvet}

  % For use with XeLaTeX
  %  \setmainfont[
  %    Path              = ./libertine/opentype/,
  %    Extension         = .otf,
  %    UprightFont = LinLibertine_R,
  %    BoldFont = LinLibertine_RZ, % Linux Libertine O Regular Semibold
  %    ItalicFont = LinLibertine_RI,
  %    BoldItalicFont = LinLibertine_RZI, % Linux Libertine O Regular Semibold Italic
  %  ]
  %  {libertine}
  %  % load font from system font
  %  \newfontfamily\libertinesystemfont{Linux Libertine O}
  
  % Use helvetica
  \renewcommand{\familydefault}{\sfdefault}
  \usepackage{helvet}
  \usepackage{sansmathfonts}
\fi

% *********
% ***** Custom Packages

% define header 
\usepackage{fancyhdr}
\usepackage{adjustbox}
\usepackage{etoolbox}
\renewcommand{\headrulewidth}{1pt}
\newcommand{\headrulecolor}[1]{\patchcmd{\headrule}{\hrule}{\color{#1}\hrule}{}{}}
\headrulecolor{CadetBlue}
\fancyheadoffset[RE,LO]{0cm}
\fancyheadoffset[LE,RO]{1.25cm}

% For recoloring the section header
\usepackage{sectsty}
\definecolor{dark-gray}{gray}{0.30}
\sectionfont{\color{dark-gray}}
\subsectionfont{\color{dark-gray}}
% \subsubsectionfont{\color{dark-gray}}

% Fancier section header
\newcommand{\hsp}{\hspace{6pt}}
\usepackage[explicit]{titlesec}
\titleformat{\section}[hang]{\Large\bfseries}{\color{dark-gray}\thesection}{6pt}{\begin{tabular}[t]{@{\color{dark-gray}\vrule width 2pt}>{\hsp}l}#1\end{tabular}}
\titleformat{\subsection}[hang]{\large\bfseries}{\color{dark-gray}\thesubsection}{6pt}{\begin{tabular}[t]{@{\color{dark-gray}\vrule width 2pt}>{\hsp}l}#1\end{tabular}}



% **** Algorithms and Pseudocode 

%\usepackage{algpseudocode}


% **** Captions and Hyperreferencing / URL 

% Captions: This makes captions of figures use a boldfaced small font.
\usepackage[font=small,labelfont=bf]{caption}
\DeclareCaptionLabelFormat{adja-page}{\hrulefill\\#1 #2 \emph{(previous page)}}

%\RequirePackage[labelsep=space,tableposition=top]{caption}
\renewcommand{\figurename}{Fig.} %to support older versions of captions.sty

\usepackage{hyperref}
\hypersetup{
    colorlinks=true,
    linkcolor=black,
    filecolor=black,      
    urlcolor=blue,
}


% *************************** Graphics and figures *****************************

%\usepackage{rotating}
\usepackage{wrapfig}

% Uncomment the following two lines to force Latex to place the figure.
% Use [H] when including graphics. Note 'H' instead of 'h'
%\usepackage{float}
%\restylefloat{figure}

% Subcaption package is also available in the sty folder you can use that by
% uncommenting the following line
% This is for people stuck with older versions of texlive
%\usepackage{sty/caption/subcaption}
\usepackage{subcaption}

% ********************************** Tables ************************************
\usepackage{booktabs} % For professional looking tables
\usepackage{multirow}

%\usepackage{multicol}
%\usepackage{longtable}
%\usepackage{tabularx}


% *********************************** SI Units *********************************
\usepackage{siunitx} % use this package module for SI units
\usepackage{textcomp} % for trademark


% ******************************* Line Spacing *********************************

% Choose linespacing as appropriate. Default is one-half line spacing as per the
% University guidelines

% \doublespacing
% \onehalfspacing
% \singlespacing


% ************************ Formatting / Footnote *******************************

% Don't break enumeration (etc.) across pages in an ugly manner (default 10000)
%\clubpenalty=500
%\widowpenalty=500

%\usepackage[perpage]{footmisc} %Range of footnote options

% Nicer heading
\renewcommand{\chaptername}{}
\usepackage[Lenny]{fncychap}


% *****************************************************************************
% *************************** Bibliography  and References ********************

%\usepackage{cleveref} %Referencing without need to explicitly state fig /table

% Add `custombib' in the document class option to use this section
\ifuseCustomBib
   \RequirePackage[square, sort, numbers, authoryear]{natbib} % CustomBib

% If you would like to use biblatex for your reference management, as opposed to the default `natbibpackage` pass the option `custombib` in the document class. Comment out the previous line to make sure you don't load the natbib package. Uncomment the following lines and specify the location of references.bib file

%\RequirePackage[backend=biber, style=numeric-comp, citestyle=numeric, sorting=nty, natbib=true]{biblatex}
%\addbibresource{References/references} %Location of references.bib only for biblatex, Do not omit the .bib extension from the filename.

\fi

% changes the default name `Bibliography` -> `References'
\renewcommand{\bibname}{References}


% ******************************************************************************
% ************************* User Defined Commands ******************************
% ******************************************************************************

% **** To change the name of Table of Contents / LOF and LOT 
\usepackage{tocloft}
\setlength\cftaftertoctitleskip{0pt}
\setlength\cftafterloftitleskip{0pt}
\setlength\cftafterlottitleskip{0pt}
\renewcommand\cfttoctitlefont{\LARGE\bfseries\scshape\color{CadetBlue}}
\renewcommand\cftloftitlefont{\LARGE\bfseries\scshape\color{CadetBlue}}
\renewcommand\cftlottitlefont{\LARGE\bfseries\scshape\color{CadetBlue}}

%\renewcommand{\glossarysection}[2][\theglstoctitle]{%
%  \def\theglstoctitle{#2}%
%  \vspace{\cftbeforelottitleskip}%
%  \par\noindent
%  \addcontentsline{toc}{chapter}{\numberline{}#1}%
%}

%\renewcommand{\contentsname}{\textcolor{Mahogany}{\LARGE \textbf{\textsc{Table of contents}}}}
%\renewcommand{\listfigurename}{\textcolor{Mahogany}{\LARGE \textbf{\textsc{List of figures}}}}
%\renewcommand{\listtablename}{\textcolor{Mahogany}{\LARGE \textbf{\textsc{List of tables}}}}

% Math abbreviations
\newcommand{\iid}{\stackrel{\mbox{iid}}{\sim}}
\newcommand{\ind}{\stackrel{\mbox{ind}}{\sim}}
\newcommand{\Lagr}{\mathcal{L}}
\newcommand{\plus}{\textsuperscript{+}}
\usepackage{tikz}
\def\checkmark{\tikz\fill[scale=0.4](0,.35) -- (.25,0) -- (1,.7) -- (.25,.15) -- cycle;} 



% ********************** TOC depth and numbering depth *************************

\setcounter{secnumdepth}{2}
\setcounter{tocdepth}{2}


% ******************************* Nomenclature 
%\usepackage{nomencl}
%\renewcommand{\nomname}{Symbols}
%\makenomenclature

\usepackage[acronym, nopostdot, nonumberlist, style=super]{glossaries}
\setlength{\glsdescwidth}{5in}

\makeglossaries
\glsdisablehyper
\renewcommand{\glsnamefont}[1]{\textbf{#1}}

% ********************************* Appendix 

% The default value of both \appendixtocname and \appendixpagename is `Appendices'. These names can all be changed via:

%\renewcommand{\appendixtocname}{List of appendices}
%\renewcommand{\appendixname}{Appndx}

% *********************** Configure Draft Mode **********************************

% Uncomment to disable figures in `draft'
%\setkeys{Gin}{draft=true}  % set draft to false to enable figures in `draft'

% These options are active only during the draft mode
% Default text is "Draft"
%\SetDraftText{DRAFT}

% Default Watermark location is top. Location (top/bottom)
%\SetDraftWMPosition{bottom}

% Draft Version - default is v1.0
%\SetDraftVersion{v1.1}

% Draft Text grayscale value (should be between 0-black and 1-white)
% Default value is 0.75
%\SetDraftGrayScale{0.8}


% ******************************** Todo Notes **********************************

% Define colour for todo notes
\newcommand{\todo}[1]{{\color{red}{#1}}}
% Colour for corrections
\newcommand{\cor}[1]{{\color{blue}{#1}}}

% *****************************************************************************
% ******************* Better enumeration my MB*************
\usepackage{enumitem}

% **********************************
% Additional packages
\usepackage{xcolor}
\definecolor{green}{RGB}{32,157,148}
\usepackage{amsmath,amsfonts,amssymb,amsthm,nccmath,systeme} % For math equations, theorems, symbols, etc
\DeclareMathAlphabet{\mathcal}{OMS}{cmsy}{m}{n} % reset mathcal to normal
\usepackage{bm} % bold math 
\usepackage{textgreek} % greek letters in text

%----------------------------------------------------------------------------------------
%	THEOREM STYLES
%----------------------------------------------------------------------------------------

%%%%%%%%%%%%%%%%%%%%%%%%%%%%%%%%%%%%%%%%%%%%%%%%%%%%%%%%%
%%% dedicated to boxed/framed environements %%%%%%%%%%%%%
%%%%%%%%%%%%%%%%%%%%%%%%%%%%%%%%%%%%%%%%%%%%%%%%%%%%%%%%%

\newtheoremstyle{greennumbox}% % Theorem style name
{0pt}% Space above
{0pt}% Space below
{\small}% % Body font
{}% Indent amount
{\small\bf\sffamily\color{green}}% % Theorem head font
{\;}% Punctuation after theorem head
{0.25em}% Space after theorem head
{\small\sffamily\color{green}\thmname{}\nobreakspace} 
\renewcommand{\qedsymbol}{$\blacksquare$}% Optional qed square

\newtheoremstyle{blacknumbox} % Theorem style name
{0pt}% Space above
{0pt}% Space below
{\normalfont}% Body font
{}% Indent amount
{\small\bf\sffamily}% Theorem head font
{}% Punctuation after theorem head
{0.25em}% Space after theorem head
{\small\sffamily\thmname{}\nobreakspace}

\newtheoremstyle{bluenumbox} % Theorem style name
{0pt}% Space above
{0pt}% Space below
{\normalfont}% Body font
{}% Indent amount
{\small\bf\sffamily\color{blue}}% Theorem head font
{}% Punctuation after theorem head
{0.25em}% Space after theorem head
{\small\sffamily\thmname{}\nobreakspace}

% Defines the theorem text style for each type of theorem to one of the three styles above
\theoremstyle{greennumbox}
\newtheorem{abstractT}{}
\theoremstyle{bluenumbox}
\newtheorem{boxT}{}
\theoremstyle{blacknumbox}
\newtheorem{commentT}{}


% Package to create coloured boxes
\RequirePackage[framemethod=default]{mdframed}

% Exercise box	  
\newmdenv[skipabove=7pt,
skipbelow=7pt,
rightline=false,
leftline=true,
topline=false,
bottomline=false,
backgroundcolor=green!10,
linecolor=green,
innerleftmargin=5pt,
innerrightmargin=5pt,
innertopmargin=5pt,
innerbottommargin=5pt,
leftmargin=0cm,
rightmargin=0cm,
linewidth=4pt]{eBox}

\definecolor{cornflowerblue}{RGB}{114,133,165}

\newmdenv[skipabove=10pt,
skipbelow=7pt,
rightline=false,
leftline=true,
topline=false,
bottomline=false,
backgroundcolor=cornflowerblue!10,
linecolor=cornflowerblue,
innerleftmargin=5pt,
innerrightmargin=5pt,
innertopmargin=5pt,
innerbottommargin=5pt,
leftmargin=0cm,
rightmargin=0cm,
linewidth=4pt
innerbottommargin=5pt]{bBox}

\newmdenv[skipabove=7pt,
skipbelow=7pt,
rightline=false,
leftline=true,
topline=false,
bottomline=false,
linecolor=gray,
backgroundcolor=black!5,
innerleftmargin=5pt,
innerrightmargin=5pt,
innertopmargin=5pt,
leftmargin=0cm,
rightmargin=0cm,
linewidth=4pt,
innerbottommargin=5pt]{cBox}

% Creates an environment for each type of theorem and assigns it a theorem text style from the "Theorem Styles" section above and a colored box from above

% \newenvironment{Abstract}{\begin{eBox}\begin{abstractT}}{\hfill{\color{green}\tiny\ensuremath{\blacksquare}}\end{abstractT}\end{eBox}}
\newenvironment{Abstract}{\begin{eBox}\begin{abstractT}}{\end{abstractT}\end{eBox}}
\newenvironment{Comment}{\begin{cBox}\begin{commentT}}{\end{commentT}\end{cBox}}	\newenvironment{Comment2}{\begin{bBox}\begin{boxT}}{\end{boxT}\end{bBox}}
