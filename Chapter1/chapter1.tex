\chapter{Introduction}  %Title of the First Chapter

The observation that human traits are heritable is evident, often visible by eye. 
Every one of us has been told that they have their mother’s eyes, their father’s height, their grandfather’s nose, etc. 
Similarly, many diseases “run in the family”: for example diabetes, or some types of breast cancer, are recurrent from generation to generation. 
In fact, for some conditions, family history can be one of the most reliable diagnostic tools. 
Describing the mechanisms by which we acquire traits, and the extent to which traits are heritable is at the core of the science we now call genetics, and is a question that has occupied scientists for years.

%********************************** %First Section  **************************************
\section{Road to quantitative genetics}  %Section - 1.1 

%********************************** % 1.1.1  **************************************
\subsection{From sweet peas to the double helix} %Section - 1.1.1 
\subsubsection{Principles of (Mendelian) Inheritance}

In On the Origin of Species (Darwin, 1859), Charles Darwin proposed the theory of evolution, which is based on the assumption that natural variation between individuals provides differential reproductive advantages, and that this variation can be inherited from one generation to the next. 
This theory nicely explains the adaptation of a species to its environment and the consequent development of new species, yet the mechanisms by which such variation occurs and the modes of inheritance were not described. 
In the words of Darwin himself in On the Origin of Species: “Our ignorance of the laws of variation is profound” and “The laws governing inheritance are quite unknown”.
\vspace{4mm}
Around the same time, specifically in 1853, the man who is now universally considered to be the father of genetics began conducting experiments to tackle this problem exactly. 
He was actually not a scientist but a friar, by the name of Gregor Mendel, and he was going to perform the now infamous experiments on inheritance in peas. 
At St. Thomas’ abbey, in Brno, Moravia (then part of the Austro-Hungarian empire), Mendel meticulously studied seven different traits of the plants (plant height, pod shape and color, seed shape and color, and flower position and color), each of which segregated on one of the plant's seven chromosomes. 
For example, seeds were either yellow or green, wrinkled or round. For seven years, Mendel followed generations and generations of pea plants and noted that some traits occurred far more often than others. 
For example, when crossing a plant with round seeds and one with wrinkled seeds, the offspring always had round seeds: Mendel called the round seed trait the “dominant” trait. 
However, the wrinkled seed trait that had seemingly vanished in the first filial generation would appear again, in the second generation in a 1 wrinkle seed plant to 3  round seed plants ratio. 
Somehow, this “recessive” trait was being passed on, remaining hidden when overpowered by the dominant trait, but not forgotten. 
In 1866, Mendel published his Laws of Inheritance in “Experiments in Plant Hybridization” in a rather unknown journal (Mendel, 1866), where he describes his experiments and results. 
The publication received almost no attention, but Mendel's Laws of Inheritance described therein will build the foundations of modern genetics.
\vspace{4mm}
Mendel and Darwin never met, and remained unaware of each other’s theories until their deaths. 
Mendel’s research remained completely unknown for decades, until at the turn of the century in 1900 four botanists (Austrian Erich von Tschermak, Dutchman Hugo de Vries, German Carl Correns and American William Jasper Spillman) independently re-discovered his work and validated his findings, officially beginning the modern age of genetics. 
\vspace{4mm}
An important step towards reconciling Darwin’s theory of evolution with Mendel’s laws of inheritance was made in 1902, when Theodor Boveri showed, in sea urchin, that different chromosomes contained different hereditary material and that organisms required a full set of chromosomes to function. 
In 1903, Walter Sutton published a paper proposing how these principles, together with the random segregation of paternal and maternal chromosomes during gamete formation (which he studied in grasshoppers) could form the molecular basis for Mendel’s Laws of Inheritance (Sutton, 1903). 
Importantly, he also noted how the number of traits was much larger than the number of chromosomes, which meant that some traits had to be located on the same chromosome and be transmitted together.

\subsubsection{Genetic Linkage and the birth of modern genetics}

In 1908, American geneticist Thomas Hunt Morgan set out to confirm (or disprove) Mendel’s theories using a model organism that generates new offspring much quicker than pea plants. 
It was Drosophila Melanogaster, the fruit fly. 
In his famous “fly room” at Columbia University, thousands of experiments were performed on flies. 
Flies with known phenotypes (such as red eyes) were put in jars to mate, and the traits of the progeny were recorded. 
Through the key observations that some traits appear to be sex-linked and that some other traits were co-occurring more often than expected by chance, Morgan theorised that “markers” responsible for particular traits were positioned on chromosomes, like beads on a string.
These markers, or genes, when close together on a chromosome were more likely to be passed on to the next generation. 
Morgan had described the concept of genetic linkage and essentially hypothesized the phenomenon of crossing over (exchange of paternal and maternal chromosomal material during meiosis; Morgan, 1911). 
In 1913 his brilliant student, Alfred Sturtevant gathered all the data collected until then and developed the first genetic map, showing the position of the fruit fly’s known genes relative to each other in terms of recombination frequency (Sturtevant, 1913). 
Sturtevant will go on to call the unit of genetic linkage a “centimorgan”, in honour of his mentor.
\vspace{4mm}
The Mendelian-chromosome theory, first proposed by Boveri and Sutton (Sutton, 1903) and then elaborated and expanded by Morgan and his students (Morgan et al., 1915) described chromosomes as the (paired) units of heredity that Mendel had described in his laws, and was widely accepted by scientists by the 1930s. 
In 1933, Morgan received the Nobel Prize in Physiology or Medicine “for his discoveries concerning the role played by the chromosome in heredity”.
\vspace{4mm}
However, the mechanisms of heredity and the physical molecule responsible for it was still unknown. 
The concept of “gene” existed, but it was an abstract entity. 
Most people in fact believed that proteins were the carriers of genetic material. 
In 1943, Erwin  Schrödinger, an Austrian-Irish physicist better known for his contributions to quantum mechanics, published “What is Life?” where he introduced the idea that genetic material may be stored as some sort of a “code”, a concept borrowed from Information Theory. 
He had provided a theoretical physical description of the mechanism of “storage” of genetic material. 
\vspace{4mm}
Still, as most scientists at the time, Schrödinger bet on proteins as the responsible molecule. 
The other candidate, deoxyribonucleic acid (DNA) had been referred to as the “stupid molecule”, a molecule with a chemical structure far too simple to be able to explain the complexity of life. 
In fact, DNA consists of only four building blocks, often referred to by their initials: adenine (A), thymine (T), cytosine (C), guanine (G).
\vspace{4mm}
Proteins remained the most likely responsible for carrying genetic information until 1944, when Oswald Avery and colleagues at the Rockefeller Institute in New York demonstrated experimentally that it had to be DNA.
Avery, along with his co-workers Colin MacLeod and Maclyn McCarthy performed an experiment in Streptococcus pneumoniae, where he removed various organic compounds from bacteria, and observed whether the bacteria could still transform. 
Only upon treating the bacteria with an enzyme that removed DNA, the bacteria stopped transforming.

\subsubsection{The double helix}

The question of the physical structure of the DNA remained unsolved until in 1953 a team of remarkable scientists proposed one structure. 
Key members of this team were Francis Crick, James Watson, Rosalind Franklin, Maurice Wilkins and Erwin Chargaff. 
Jim Watson had had a fascination for the structure of DNA, and had been studying it for years, starting in his native Chicago, then during his PhD in Indiana (under the supervision of Italian future Nobel Prize laureate Salvador Luria), to then end up at the Lucy Cavendish laboratory, then directed by Australian-born British X-ray crystallographer Sir Lawrence Bragg, in Cambridge, United Kingdom. 
There, he met Francis Crick, 12 years his senior, a brilliant British physicist turned biologist. 
Crick had taught himself the mathematical theory of X-ray crystallography and had worked on determining the most stable helical conformation of amino acid chains in proteins, the alpha helix, only to be beaten to its solution by American chemist Linus Pauling. 
Watson and Crick set out to obtain a model for the structure of DNA, building on Crick’s experience and rigor, and Watson’s intuition. 
Friend and collaborator of the pair was Maurice Wilkins, New Zealand-born British physicist at King’s College London, who had extensively studied X-ray diffraction patterns. 
His colleague, Rosalind Franklin had perfected the technique to produce X-ray crystallography images of the DNA and instructed her assistant, Raymond Gosling, to take the most precise of them to date, the now famous “Photo 51”. 
To Watson and Crick’s eyes, Photo 51 (which Wilkins had shared without Franklin knowing), looked without a doubt like the shadow that a double helix would leave. 
The last piece of the puzzle came from a discovery that Austro-Hungarian Erwin Chargaff made, at Columbia University. 
He observed that globally the amounts of As and Ts in DNA were roughly the same, as were the amounts of Cs and Gs. 
This provided the idea that bases would be paired up and facing inwards in the double helix, As with Ts, Cs with Gs, ensuring that the covalent bonds would be always of the same length, keeping the helix stable. 
In 1953, Watson and Crick published “Molecular structure of nucleic acids” (Watson and Crick, 1953). 
Their work showed how the four nucleotide bases (A, T, C, G) formed “two helical chains each coiled round the same axis” (Watson and Crick, 1953) spelling out what Crick called “the secret of life.” 
For this discovery, Crick, Watson and Wilkins won the 1962 Nobel Prize in Physiology or Medicine "for their discoveries concerning the molecular structure of nucleic acids and its significance for information transfer in living material". 
Rosalind Franklin, who had played a critical role in the discovery had died four years prior of ovarian cancer, and her contribution went largely unrecognized.

\vspace{4mm}
It is DNA that contains the instructions of life, that encodes how we function and how we look. But each individual’s copy of DNA must not be identical, there must be some differences that explain observable physical differences. 

%********************************** % 1.1.2  **************************************
\subsection{Mendelian and complex diseases} %Section - 1.1.2 

In the early 1900s, the genetic basis of traits and diseases was fiercely debated between the Mendelians and the biometricians, in what is commonly referred to as the ‘Biometric-Mendelian debate’ [REF]. The biometricians followed theories first put forward by Francis Galton [REF] and believed that Mendelian genetics could not explain the continuous distribution of many traits observed in humans [REF] 
This debate was resolved in a seminal 1918 paper by R.A. Fisher, who showed that, if many genes affect a trait, then the random sampling of alleles at each gene produces a continuous, normally distributed phenotype in the population (Fisher, 1918). As the number of genes grows very large, the contribution of each gene becomes correspondingly smaller, leading in the limit to Fisher’s famous ‘infinitesimal model’ (Barton et al., 2016).

It is now widely accepted that both modes of inheritance exist, such that some traits are classified as Mendelian and others as complex.

Mendelian traits are also referred to as monogenic disorders, which as the name suggests, are driven by mutations within a single gene. Examples of such traits include X-linked muscular dystrophies, cystic fibrosis, Fanconi anaemia, the classic form of Ehlers-Danlos syndrome and phenylketonuria11. Phenylketonuria is an example of a monogenic disorder that only manifests under specific environmental conditions; specifically the phenotype only occurs when there are both mutations within the PAH gene and when an individual is exposed to phenylalanine (naturally occurring in dietary proteins)REF. Mendelian traits are typically rare in the general population but often cluster in families [REF]. 

In comparison, complex traits are typically common in the general population. They are driven by a combination of multiple genetic risk factors across the genome (thus they are often referred to as polygenic traits), environmental risk factors, as well as interaction effects between genetic variants and environmental exposures. The combined contribution of the genetic and environmental factors, results in a continuous range of phenotypic values; hence complex traits are sometimes referred to as quantitative traits. Such complex traits include height, weight and blood pressure. However, for some complex traits, the combination of genetic and environmental factors can be viewed as a predisposition measure that when combined with a suitable penetrance (defined here as the probability of having a disease given the predisposition score) mapping function results in an observed binary outcome (i. e. either diseased or not). Examples of complex traits with binary outcomes include heart disease, type 2 diabetes, schizophrenia, asthma and cancer. 

It is now recognised that this binary classification of phenotypes as Mendelian or complex is an oversimplification and that a continuum spanning from monogenic to polygenic disorders exists. Traits that bridge the two are often referred to as oligogenic. This additional classification stemmed from the fact that not all individuals with known Mendelian mutations presented with the expected phenotype, termed incomplete penetrance, suggesting the presence of a few modifier genes. Examples include phenylketonuria, cystic fibrosis and Hirschsprung disease, traits that were classically considered to be Mendelian.
