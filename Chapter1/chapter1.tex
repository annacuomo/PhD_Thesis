\chapter{Introduction}  %Title of the First Chapter

The observation that human traits are heritable is evident, often visible by eye. 
Every one of us has been told that they have their mother’s eyes, their father’s height, their grandfather’s nose, etc. 
Similarly, many diseases “run in the family”: for example diabetes, or some types of breast cancer, are recurrent from generation to generation. 
In fact, for some conditions, family history can be one of the most reliable diagnostic tools. 
Describing the mechanisms by which we acquire traits, and the extent to which traits are heritable is at the core of the science we now call genetics, and is a question that has occupied scientists for years.

%********************************** %First Section  **************************************
\section{From peas to GWAS}  %Section - 1.1 

%********************************** % 1.1.1  **************************************
\subsection{Principles of (Mendelian) Inheritance} %Section - 1.1.1 

In On the Origin of Species (Darwin, 1859), Charles Darwin proposed the theory of evolution, which is based on the assumption that natural variation between individuals provides differential reproductive advantages, and that this variation can be inherited from one generation to the next. 
This theory nicely explains the adaptation of a species to its environment and the consequent development of new species, yet the mechanisms by which such variation occurs and the modes of inheritance were not described. 
In the words of Darwin himself in On the Origin of Species: “Our ignorance of the laws of variation is profound” and “The laws governing inheritance are quite unknown”.\\

Around the same time, specifically in 1853, the man who is now universally considered to be the father of genetics began conducting experiments to tackle this problem exactly. 
He was actually not a scientist but a friar, by the name of Gregor Mendel, and he was going to perform the now infamous experiments on inheritance in peas. 
At St. Thomas’ abbey, in Brno, Moravia (then part of the Austro-Hungarian empire), Mendel meticulously studied seven different traits of the plants (plant height, pod shape and color, seed shape and color, and flower position and color), each of which segregated on one of the plant's seven chromosomes. 
For example, seeds were either yellow or green, wrinkled or round. For seven years, Mendel followed generations and generations of pea plants and noted that some traits occurred far more often than others. 
For example, when crossing a plant with round seeds and one with wrinkled seeds, the offspring (F1) always had round seeds: Mendel called the round seed trait the “dominant” trait. However, the wrinkled seed trait that had seemingly vanished in the first filial generation would appear again, in the second generation (F2) in a 1 wrinkle seed plant to 3 round seed plants ratio. Somehow, this “recessive” trait was being passed on, remaining hidden when overpowered by the dominant trait, but not forgotten. 
In 1866, Mendel published his experiments and results in “Experiments in Plant Hybridization” (Mendel, 1866). 
In it, he proposes what will be called the Mendelian Laws of Inheritance: i) the Law of Independent Segregation (every individual contains two factors for each trait one of which is passed on to its offspring at random), ii) the Law of Independent Assortment (traits are inherited independently of each other) and iii) the Law of Dominance (recessive alleles will be masked by dominant alleles and the trait corresponding to the dominant allele will be observed) [Mendel, 1866]. 
The publication received almost no attention, but his Laws of Inheritance described therein will build the foundations of modern genetics.\\

Mendel and Darwin never met, and remained unaware of each other’s theories until their deaths. Mendel’s research remained completely unknown for decades, until at the turn of the century in 1900 four botanists (Austrian Erich von Tschermak, Dutchman Hugo de Vries, German Carl Correns and American William Jasper Spillman) independently re-discovered his work and validated his findings, officially beginning the modern age of genetics.\\ 

Around the same time, the British geneticist William Bateson set out to make Mendel’s work accessible to the scientists not proficient in Mendel’s native language German.
He translated Mendel’s original papers on the Laws of Inheritance into English and published them which allowed Mendel’s work to become known in the greater scientific world [Bateson, 1909; Keynes \& Cox, 2008], more than 40 years after their original publication.\\ 

An important step towards reconciling Darwin’s theory of evolution with Mendel’s laws of inheritance was made in 1902, when Theodor Boveri showed, in sea urchin, that different chromosomes contained different hereditary material and that organisms required a full set of chromosomes to function. 
In 1903, Walter Sutton published a paper proposing how these principles, together with the random segregation of paternal and maternal chromosomes during gamete formation (which he studied in grasshoppers) could form the molecular basis for Mendel’s Laws of Inheritance (Sutton, 1903). 
Importantly, he also noted how the number of traits was much larger than the number of chromosomes, which meant that some traits had to be located on the same chromosome and be transmitted together.\\

\subsection{Genetic Linkage and the birth of modern genetics} %Section - 1.1.2 

In 1908, American geneticist Thomas Hunt Morgan set out to confirm (or disprove) Mendel’s theories using a model organism that generates new offspring much quicker than pea plants. 
It was Drosophila Melanogaster, the fruit fly. 
In his famous “fly room” at Columbia University, thousands of experiments were performed on flies. 
Flies with known phenotypes (such as red eyes) were put in jars to mate, and the traits of the progeny were recorded. 
Through the key observations that some traits appear to be sex-linked and that some other traits were co-occurring more often than expected by chance, Morgan theorised that “markers” responsible for particular traits were positioned on chromosomes, like beads on a string. 
These markers, or genes, when close together on a chromosome were more likely to be passed on to the next generation. 
Morgan had described the concept of genetic linkage and essentially hypothesized the phenomenon of crossing over (exchange of paternal and maternal chromosomal material during meiosis; Morgan, 1911). 
In 1913 his brilliant student, Alfred Sturtevant gathered all the data collected until then and developed the first genetic map, showing the position of the fruit fly’s known genes relative to each other in terms of recombination frequency (Sturtevant, 1913). 
Sturtevant will go on to call the unit of genetic linkage a “centimorgan”, in honour of his mentor.\\

The Mendelian-chromosome theory, first proposed by Boveri and Sutton (Sutton, 1903) and then elaborated and expanded by Morgan and his students (Morgan et al., 1915) described chromosomes as the (paired) units of heredity that Mendel had described in his laws, and was widely accepted by scientists by the 1930s. 
In 1933, Morgan received the Nobel Prize in Physiology or Medicine “for his discoveries concerning the role played by the chromosome in heredity”.\\

However, the mechanisms of heredity and the physical molecule responsible for it was still unknown. 
The concept of “gene” existed, but it was an abstract entity. 
Most people in fact believed that proteins were the carriers of genetic material. 
In 1943, Erwin  Schrödinger, an Austrian-Irish physicist better known for his contributions to quantum mechanics, published “What is Life?” where he introduced the idea that genetic material may be stored as some sort of a “code”, a concept borrowed from Information Theory. 
He had provided a theoretical physical description of the mechanism of “storage” of genetic material.\\ 

Still, as most scientists at the time, Schrödinger bet on proteins as the responsible molecule. 
The other candidate, deoxyribonucleic acid (DNA) had been referred to as the “stupid molecule”, a molecule with a chemical structure far too simple to be able to explain the complexity of life. 
In fact, DNA consists of only four building blocks, often referred to by their initials: adenine (A), thymine (T), cytosine (C), guanine (G).\\

Proteins remained the most likely responsible for carrying genetic information until 1944, when Oswald Avery and colleagues at the Rockefeller Institute in New York demonstrated experimentally that it had to be DNA. 
Avery, along with his co-workers Colin MacLeod and Maclyn McCarthy performed an experiment in Streptococcus pneumoniae, where he removed various organic compounds from bacteria, and observed whether the bacteria could still transform. 
Only upon treating the bacteria with an enzyme that removed DNA, the bacteria stopped transforming.

\subsection{The double helix} %Section - 1.1.3

The question of the physical structure of the DNA remained unsolved until in 1953 a team of remarkable scientists proposed one structure. 
Key members of this team were Francis Crick, James Watson, Rosalind Franklin, Maurice Wilkins and Erwin Chargaff. 
Jim Watson had had a fascination for the structure of DNA, and had been studying it for years, starting in his native Chicago, then during his PhD in Indiana (under the supervision of Italian future Nobel Prize laureate Salvador Luria), to then end up at the Lucy Cavendish laboratory, then directed by Australian-born British X-ray crystallographer Sir Lawrence Bragg, in Cambridge, United Kingdom. 
There, he met Francis Crick, 12 years his senior, a brilliant British physicist turned biologist. 
Crick had taught himself the mathematical theory of X-ray crystallography and had worked on determining the most stable helical conformation of amino acid chains in proteins, the alpha helix, only to be beaten to its solution by American chemist Linus Pauling. 
Watson and Crick set out to obtain a model for the structure of DNA, building on Crick’s experience and rigor, and Watson’s intuition. 
Friend and collaborator of the pair was Maurice Wilkins, New Zealand-born British physicist at King’s College London, who had extensively studied X-ray diffraction patterns. 
His colleague, Rosalind Franklin had perfected the technique to produce X-ray crystallography images of the DNA and instructed her assistant, Raymond Gosling, to take the most precise of them to date, the now famous “Photo 51”. 
To Watson and Crick’s eyes, Photo 51 (which Wilkins had shared without Franklin knowing), looked without a doubt like the shadow that a helix would leave. The last piece of the puzzle came from a discovery that Austro-Hungarian Erwin Chargaff made, at Columbia University. He observed that globally the amounts of As and Ts in DNA were roughly the same, as were the amounts of Cs and Gs. 
This provided the idea that bases would be paired up and facing inwards in the double helix, As with Ts, Cs with Gs, ensuring that the covalent bonds would be always of the same length, keeping the helix stable. 
In 1953, Watson and Crick published “Molecular structure of nucleic acids” (Watson and Crick, 1953). 
Their work showed how the four nucleotide bases (A, T, C, G) formed “two helical chains each coiled round the same axis” (Watson and Crick, 1953) spelling out what Crick called “the secret of life.” 
For this discovery, Crick, Watson and Wilkins won the 1962 Nobel Prize in Physiology or Medicine “for their discoveries concerning the molecular structure of nucleic acids and its significance for information transfer in living material”. 
Rosalind Franklin, who had played a critical role in the discovery had died four years prior of ovarian cancer, and her contribution went largely unrecognized.


\subsection{Biometrics} %Section - 1.1.4
While Mendel was studying the inheritance of traits in peas, Boveri studied sea urchins, Morgan fruit flies, Avery bacteria, and long before Crick, Watson and Franklin proposed a structure for DNA (in squid), ever since Darwin’s theories, others were trying to quantify inheritance in the context of human traits.\\ 

One investigator among them was Francis Galton, a half-cousin of Darwin’s, who was interested in mathematically describing and analysing Darwin’s evolutionary concepts. 
He was particularly fascinated by the question of how evolution applied to humanity and how its effects could be used to improve the human race. 
To this end, Galton applied himself to the study of biometrics, trying to measure and estimate the heritability of human traits such as height and mental capabilities.\\ 

Linked to these efforts, and on a less honourable note, Galton was also the founder of eugenics, a theory for which genetics should be used as a tool to force evolution’s hand by encouraging mating of individuals considered to have especially desirable qualities and on the other hand by eliminating or preventing reproduction of individuals considered faulted. Eugenics theories are linked to one of the most horrifying pages of human history, motivating forced sterilizations of the “unfit” in the United States in the 1920s and 1930s and of course having being used as justification for the racial policies of Nazi Germany.\\ 

Nevertheless, some of the concepts and methods he developed during these studies are still fundamental to genetics today (Galton, 1909). 
These include the concepts of correlation, regression toward the mean and the regression line, which Galton used to compare the heights of children to those of their parents. Galton’s protégé was the mathematician Karl Pearson, who worked together with Galton to make several more important contributions to statistics. Among others, he introduced the concepts of the p-value and the chi squared test (Pearson, 1900) and proposed principal component analysis (PCA, Pearson, 1901).

\subsection{Towards quantitative genetics} %Section - 1.1.5

By cross-breeding Drosophila lines and performing genetic mapping, Morgan and his students had conducted the first genotype-phenotype studies. 
Similar to Mendel and Bateson the phenotypes they observed were predominantly categorical, such as the colour and shape of seeds in pea plants or the red or white-eyed phenotype in Drosophila melanogaster. 
In contrast, biometricians like Galton and Pearson had mostly looked at continuous traits in humans, such as height, and believed that those could not be explained by Mendelian genetics.
This controversy is commonly referred to as the ‘Biometric-Mendelian debate’.\\ 

This debate was resolved by British statistician Ronald A. Fisher, who in a seminal 1918 paper showed that, if many genes affect a trait, then the random sampling of alleles at each gene produces a continuous, normally distributed phenotype in the population (Fisher, 1918). 
As the number of genes grows very large, the contribution of each gene becomes correspondingly smaller, leading in the limit to Fisher’s famous ‘‘infinitesimal model’’ (Barton et al., 2016).\\

In addition to showing that biometrics and Mendelianism are not contradictory but complementary, Fisher made several contributions to the field, outlining statistical ideas and tools still used today. 
An undergraduate student at the University in Cambridge, Fisher [1912] published his first paper On a absolute criterion for fitting frequency curves where he outlined the fundamental ideas of maximum likelihood estimation (MLE). 
He later extended on this work and by 1922, he had established the properties of the maximum likelihood estimator such as consistency and minimum variability [Fisher, 1922b] that is still used today [Hald, 1999]. 
He demonstrated the utility of maximum likelihood estimation in genetics by solving a number of equations to elucidate a genetic map of eight Drosophila melanogaster genes based on their crossing over frequencies [Fisher, 1922d].\\ 

In the same year and years to follow, he published a series of papers where he derived the distribution and significance testing of regression coefficients, correlation ratios and multiple regression coefficients [Fisher, 1922c; Fisher, 1928], an exact test for two-by-two contingency tables with small expectations (Fisher’s exact test) [Fisher, 1922a], partial correlation coefficients [Fisher, 1924b] and the variance ratio, later named after Fisher as the F statistic [Fisher, 1924a]. 

Additionally, in his 1918 work, Fisher introduced the concepts of variance (as “the square root of the mean squared error”) and analysis of variance (ANOVA). 
He employed these concepts during his appointment at Rothamsted Experimental Station where he analysed data from crop experiments that had been produced by the agricultural research institute over many decades.
For example, he investigated the effects of different types of fertiliser on wheat yield, using a data set that covered the yield of 13 differently treated plots of land over more than 60 years. 
This work resulted in his series of publications entitled Studies in Crop Variation (see for example Fisher, 1921; Fisher and Mackenzie, 1923). 

In 1930, Fisher published the book The Genetical Theory of Natural Selection where he reconciled Darwin’s evolutionary theory of natural selection and Mendel’s inheritance laws. 
He gave the first, comprehensive quantitative theory of sexual selection, evolution of recombination rates, polymorphism and many more concepts found in today’s field of population genetics [Fisher, 1930]. 
Thus, together with J.B.S. Haldane and Sewall Wright, Fisher essentially founded the field of population genetics in the 1930s.\\

\subsection{Technological advancements}
\subsubsection{DNA sequencing}
\subsection{Genotype mapping}

\subsection{Genotype-phenotype studies}
\subsubsection{linkage}
\subsubsection{GWAS}

%********************************** %Second Section  **************************************
\section{Gene expression and eQTL mapping}  %Section - 1.2 

\subsection{DNA and the central dogma}
The resolution of the DNA structure and all discoveries that lead to it brought forward an understanding of other biological concepts such as protein synthesis and enabled Francis Crick to postulate the central dogma of biology: information is transmitted from nucleic acids (DNA and RNA) to proteins, but information cannot be transmitted from a protein to DNA [Crick, 1958]. 
The deciphering of the genetic code through Nirenberg and others followed a few years later [Nirenberg \& Matthaei, 1961; Crick \& al., 1961; Matthaei \& al., 1962].\\

Today, we know that DNA gets transcribed into RNA aided by the DNA polymerase molecule, and RNA gets translated to amino-acids making up proteins in the ribosome, with the help of transfer RNA (tRNA).

\subsection{Gene regulation}
\subsection{Estimation of gene expression levels}
\subsection{Expression quantitative trait loci}

%********************************** %Third Section  **************************************
\section{Single cell RNA-seq}  %Section - 1.3 

%********************************** %Fourth Section  **************************************
\section{Human induced pluripotent stem cells}  %Section - 1.4 

%****** Box on exponential family distributions ******

\newpage

\begin{Comment}
\hspace{-2.5mm}\textbf{Box 1: Model organisms}\label{box1}\\
% \small

Model organisms are non-human species used to study biological phenomena:

\begin{itemize}
    \item Bacteria (\textit{Escherichia coli}, or \textit{E. coli})
    \item Budding Yeast (\textit{Saccharomyces cerevisiae} or \textit{S. cerevisiae})
    \item Thale cress (\textit{Arabidopsis Thaliana})
    \item Fruit fly (\textit{Drosophila melanogaster})
    \item Nematode worm (\textit{Caenorhabditis elegans} or \textit{C. elegans})
    \item Western clawed frog (\textit{Xenopus tropicalis})
    \item Zebrafish (\textit{Danio rerio})
    \item Mouse (\textit{Mus musculus})

\end{itemize}


\end{Comment}

%**************