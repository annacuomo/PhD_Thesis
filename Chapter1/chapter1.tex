\chapter{Introduction}  %Title of the First Chapter

The observation that human traits are heritable is evident, often visible by eye. Every one of us has been told that they have their mother’s eyes, their father’s height, their grandfather’s nose, etc. Similarly, many diseases “run in the family”: for example diabetes, or some types of breast cancer, are recurrent from generation to generation. In fact, for some conditions, family history can be one of the most reliable diagnostic tools. Describing the mechanisms by which we acquire traits, and the extent to which traits are heritable is at the core of the science we now call genetics, and is a question that has occupied scientists for years.

%********************************** %First Section  **************************************
\section{Road to quantitative genetics}  %Section - 1.1 

%********************************** % 1.1.1  **************************************
\subsection{From sweet peas to population genetics} %Section - 1.1.1 
In On the Origin of Species (Darwin, 1859), Charles Darwin proposed the theory of evolution, which is based on the assumption that natural variation between individuals provides differential reproductive advantages, and that this variation can be inherited from one generation to the next. This theory nicely explains the adaptation of a species to its environment and the consequent development of new species, yet the mechanisms by which such variation occurs and the modes of inheritance were not described. In the words of Darwin himself in On the Origin of Species: “Our ignorance of the laws of variation is profound” and “[t]he laws governing inheritance are quite unknown”.

Around the same time, specifically in 1853, the man who is now universally considered to be the father of genetics began conducting experiments to tackle this problem exactly. He was actually not a scientist but a friar, by the name of Gregor Mendel, and he was going to perform the now infamous experiments on inheritance in peas. Mendel meticulously studied seven different traits of the plants (plant height, pod shape and color, seed shape and color, and flower position and color), each of which segregated on one of the plant's seven chromosomes. For example, seeds were either yellow or green, wrinkled or round. In 1865, Mendel published his Laws of Inheritance in “Experiments in Plant Hybridization”, where he describes his experiments and results. The publication received almost no attention, but Mendel's Laws of Inheritance described therein had begun the era of modern genetics.

Mendel and Darwin never met, and remained unaware of each other’s theories until their deaths. 

Fast forward almost 100 years and after a series of major scientific discoveries and technological advancements followed over the next century (Figure XX) in 1953 a team of scientists proposed a structure for deoxyribonucleic acid (DNA): key members of this team were Francis Crick, James Watson, Rosalind Franklin, and Maurice Wilkins. Crick provided a mathematical theory regarding the X-ray diffraction pattern of helical structures; Watson had been studying the DNA and dedicated years to determining its structure; Franklin produced the first images of DNA using X-ray crystallography and Wilkins had studied X-ray diffraction patterns. Their work intersected, and Watson and Crick published “Molecular structure of nucleic acids” (Watson and Crick, 1953) in 1953. Their work showed how the four nucleotide bases - adenine (A), cytosine (C), guanine (G), and thymine (T) - formed “two helical chains each coiled round the same axis” (Watson and Crick, 1953) spelling out what Crick called “the secret of life.”

It is DNA that contains the instructions of life, that encodes how we function and how we look. But each individual’s copy of DNA must not be identical, there must be some differences that explain observable physical differences. 

%********************************** % 1.1.2  **************************************
\subsection{Mendelian and complex diseases} %Section - 1.1.2 

In the early 1900s, the genetic basis of traits and diseases was fiercely debated between the Mendelians and the biometricians, in what is commonly referred to as the ‘Biometric-Mendelian debate’ [REF]. The biometricians followed theories first put forward by Francis Galton [REF] and believed that Mendelian genetics could not explain the continuous distribution of many traits observed in humans [REF] 
This debate was resolved in a seminal 1918 paper by R.A. Fisher, who showed that, if many genes affect a trait, then the random sampling of alleles at each gene produces a continuous, normally distributed phenotype in the population (Fisher, 1918). As the number of genes grows very large, the contribution of each gene becomes correspondingly smaller, leading in the limit to Fisher’s famous ‘infinitesimal model’ (Barton et al., 2016).

It is now widely accepted that both modes of inheritance exist, such that some traits are classified as Mendelian and others as complex.

Mendelian traits are also referred to as monogenic disorders, which as the name suggests, are driven by mutations within a single gene. Examples of such traits include X-linked muscular dystrophies, cystic fibrosis, Fanconi anaemia, the classic form of Ehlers-Danlos syndrome and phenylketonuria11. Phenylketonuria is an example of a monogenic disorder that only manifests under specific environmental conditions; specifically the phenotype only occurs when there are both mutations within the PAH gene and when an individual is exposed to phenylalanine (naturally occurring in dietary proteins)REF. Mendelian traits are typically rare in the general population but often cluster in families [REF]. 

In comparison, complex traits are typically common in the general population. They are driven by a combination of multiple genetic risk factors across the genome (thus they are often referred to as polygenic traits), environmental risk factors, as well as interaction effects between genetic variants and environmental exposures. The combined contribution of the genetic and environmental factors, results in a continuous range of phenotypic values; hence complex traits are sometimes referred to as quantitative traits. Such complex traits include height, weight and blood pressure. However, for some complex traits, the combination of genetic and environmental factors can be viewed as a predisposition measure that when combined with a suitable penetrance (defined here as the probability of having a disease given the predisposition score) mapping function results in an observed binary outcome (i. e. either diseased or not). Examples of complex traits with binary outcomes include heart disease, type 2 diabetes, schizophrenia, asthma and cancer. 

It is now recognised that this binary classification of phenotypes as Mendelian or complex is an oversimplification and that a continuum spanning from monogenic to polygenic disorders exists. Traits that bridge the two are often referred to as oligogenic. This additional classification stemmed from the fact that not all individuals with known Mendelian mutations presented with the expected phenotype, termed incomplete penetrance, suggesting the presence of a few modifier genes. Examples include phenylketonuria, cystic fibrosis and Hirschsprung disease, traits that were classically considered to be Mendelian.
