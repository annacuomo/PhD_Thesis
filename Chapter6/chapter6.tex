%!TEX root = ../thesis.tex
%*******************************************************************************
%****************************** Sixth Chapter *********************************
%*******************************************************************************

\chapter{LMM framework for multi-environment interaction eQTL using scRNA-seq}
\label{chapter6}

% add abstract
Abstract will go here

\newpage

% \begin{Comment2}
% % \subsection{Contributions}
% \hspace{-3mm}\textbf{Contributions} I worked on this project in collaboration with Danilo Horta, Elissavet (Elsa) Kentepozidou, John Marioni and Oliver Stegle.
% In this project, I extended an existing model for identifying multivariate GxE interactions (StructLMM \cite{moore2019linear}) to the specific use case of scRNA-seq data.
% I worked out the necessary underlying maths necessary for the extension and helped re-writing the method in collaboration with Danilo. 
% The model is now a usable python package available here:
% Please also find the corresponding documentation here:

% Anna S.E. Cuomo, Danilo Horta, Elissavet Kentepozidou, Marc Jan Bonder, John C. Marioni, Oliver Stegle. 
% Title, 2020

% \end{Comment2}

% \newpage

\section{Introduction} 

As we have seen in the previous chapters, population-scale single-cell sequencing studies, which combine genotype information and single cell expression profiles across larger sets of individuals, have enabled the study of genetic effects on gene regulation at single-cell resolution. 
Seminal studies have demonstrated the feasibility of mapping expression quantitative trait loci (eQTL) using single-cell readouts, thereby replicating conventional eQTL detected using bulk gene expression profiling \cite{cuomo2020single, van2018single}, detecting cell-type specific genetic effects \cite{van2018single} as well as identifying dynamic changes of regulatory effects across lineages or cellular differentiation \cite{cuomo2020single}. 
Collectively, these efforts have demonstrated the potential of single-cell sequencing approaches to complement and generalise conventional eQTL mapping, elucidating context-specific regulation, which previously was limited to narrowly defined context such as cell types from sorted populations \cite{fairfax2012genetics} or external stimulation \cite{fairfax2014innate}.\\

Despite this potential of single-cell genetics, appropriate analysis methods are only beginning to emerge. 

On the one hand existing multi-tissue eQTL methods designed for bulk profiling, which allow for mapping eQTL that are present in single tissues or combinations of tissues, remain applicable. 
For example, …\cite{flutre2013statistical, urbut2019flexible}.
However, this requires to discretize single-cell profiles into cell types, which fails to fully leverage the advantages of single-cell readouts to detect complex gene-context interactions, in particular the ability to detect continuous changes and gradients. \\

Recently, methods to assess GxE using larger numbers of context have emerged in the context of population studies,... StructLMM…


% population-scale cohorts that combine genotyping with single cell expression profiles have fostered interest to study the effect of common genetic variants on gene regulation at single cell resolution.\\

% It has already been demonstrated that single cell \gls{eqtl} mapping can be achieved, and that known \glspl{eqtl} previously identified using bulk expression can be replicated both in iPS cells and blood \cite{cuomo2020single,van2018single}.
% Additionally, the single cell resolution allows for the interrogation of interaction eQTL, where the strength of the genetic effect of eQTL is modulated by the cells’ types and states \cite{cuomo2020single}. 
% These kinds of studies can add a layer of interpretability of the role of genetic variants on gene expression regulation. 
% Analyses of eQTL context-specificity have been so far limited to a single state variable (e.g. cell type, \cite{fairfax2012genetics} or stimulation status,\cite{fairfax2014innate} ) and individual genetic variants.\\

A recently proposed method called StructLMM (structured linear mixed model, described in 2.4.3) allows for the robust joint analysis of GxE effect of multiple environmental variables \cite{moore2019linear}.
This model can be applied, with some care, to the mapping of interaction eQTL, where genes’ single cell expression profiles are the tested outcomes, and the cells’ states and types are the environmental variables. 
The main limitation of StructLMM is that it does not allow to fully control for the intrinsic population structure of the tested samples [REF]. 
When using single cell measurements, we are in practice dealing with genetically identical replicate observations for a donor. 
The relatedness structure between samples (or cells in this context) is therefore non-trivial, as the cell measurements are not independent, as assumed by the model.\\

Here, we propose sc-StructLMM, an extension of the StructLMM model that builds on it while allowing control for confounding effects due to population structure, and is especially well suited for the mapping of multivariate single cell interaction eQTL. 
The model can handle hundreds of cell states and conditions, and it can be applied to large cohorts of hundreds of thousands of cells for hundreds of  individuals.

\newpage

\section{Model} 

Commonly used methods for eQTL mapping fit linear or linear mixed models (LMMs),  whereby a given variant is tested for association with expression changes of a target gene \cite{kilpinen2017common}. 
The effect size is assumed to be shared (persistent) across individuals, or alternatively a discrete group structure (e.g. tissues) can be accounted for {ref}. 
scStructLMM in contrast borrows concepts from the previously described StructLMM model designed for global phenotypes \cite{moore2019linear}, whereby heterogeneity in effect sizes is defined by an environmental (context) covariance matrix, which can capture arbitrary substructure in environmental states. 
scStructLMM generalizes this previous approaches by the ability to  account for two distinct sets of additive random effect components, which in particular enables modelling the intrinsic repeat structure of single-cell eQTL studies where multiple cells are sampled form the same individual. 
The model can be cast as:

\begin{equation}\label{eq:scStructLMM}
 \mathbf{y} =  \mathbf{W}\boldsymbol{\alpha} + \mathbf{g}\beta_G + \mathbf{g} \odot \boldsymbol{\beta_{GxE}} + \mathbf{e} + \mathbf{u} + \boldsymbol{\psi}, 
\end{equation}

% where $\mathbf{e} \sim \mathcal{N}\mathbf{0},\sigma_e^2 \mathbf{E}\mathbf{E}^T)$, $\mathbf{u} \sim \mathcal{N}\mathbf{0},\sigma_g^2 \mathbf{G}\mathbf{G}^T)$ and $\boldsymbol{\psi} \sim \mathcal{N}\mathbf{0},\sigma_n^2 \mathbf{I_N})$.\\

Here, $\beta_G$ denotes the effect size of a conventional persistent genetic effect component, and $\boldsymbol{\beta_{GxE}}=[\beta_{GxE_1}, .. ,\beta_{GxE_N}]^T$ is a vector of per-cell effect sizes to account for heterogeneous genetic effects, which follows a multivariate normal distribution, $\boldsymbol{\beta_{GxE}} \sim \mathcal{N}\mathbf{0},\sigma_{GxE}^2 \mathbf{E}\mathbf{E}^T)$. 
Depending on the functional form of the environmental covariance , this model can account for different types of G×E, for example, both discrete and continuous cell states and types or donor-level environmental covariates. 
The same environmental covariance is also used to account for additive environmental effects, $\mathbf{e} \sim \mathcal{N}\mathbf{0},\sigma_e^2 \mathbf{E}\mathbf{E}^T)$ \cite{moore2019linear}. 
Finally, additive global genetic effects are accounted using a classical genetic relatedness matrix (GRM) \cite{purcell2007plink} also called kinship matrix (K), $\mathbf{u} \sim \mathcal{N}\mathbf{0},\sigma_g^2 \mathbf{G}\mathbf{G}^T)$, appropriately expanded to reflect the repeated structure of multiple cells derived from the same donors; finally the noise term is gaussian and i.i.d. $\boldsymbol{\psi} \sim \mathcal{N}\mathbf{0},\sigma_n^2 \mathbf{I_N})$.\\

Because we use Rao's Score test, we only need to evaluate the likelihood under $H_0$:

\begin{equation}\label{eq:scStructLMM_H0}
 \mathbf{y}|H_0 =  \mathbf{W}\boldsymbol{\alpha} + \mathbf{g}\beta_G + \mathbf{e} + \mathbf{u} + \boldsymbol{\psi} 
\end{equation}

\begin{equation}\label{eq:scStructLMM_H0_MVN}
 \mathbf{y}|H_0 \sim \mathcal{N} \mathbf{W}\boldsymbol{\alpha} + \mathbf{g}\beta_G, \sigma_e^2 \mathbf{E}\mathbf{E}^T + \sigma_g^2 \mathbf{G}\mathbf{G}^T+ \sigma_n^2 \mathbf{I} )
\end{equation}

Now to be able to use the trick described in Section 2.3.3, we need to write the Covariance matrix in the form $\sigma^2(\mathbf{M}+\delta\mathbf{I})$ as in \eqref{eq:fast_lmm_full_covariance}.
To do so, we introduce a weight parameter $\rho_1$ such that:

Covariance under the null, $\mathbf{K}_0 = Cov(\mathbf{y} | H_0)$ is:

\begin{equation}
\begin{split}
    \mathbf{K}_0 = \sigma_e^2 \mathbf{E}\mathbf{E}^T + \sigma_g^2 \mathbf{G}\mathbf{G}^T+ \sigma_n^2 \mathbf{I} =\\
    \sigma_k^2[\rho_1\mathbf{E}\mathbf{E}^T + (1-\rho_1) \mathbf{G}\mathbf{G}^T] + \sigma_n^2 \mathbf{I} =\\ \sigma_k^2\{\boldsymbol{\Sigma}(\rho_1) + \delta_1 \mathbf{I}\}
\end{split}
\end{equation}

Where $\sigma_k^2\rho_1 = \sigma_e^2$ and $\sigma_k^2(1-\rho_1) = \sigma_g^2$;
$\delta_1 = \sigma_n^2/\sigma_k^2$ and $\boldsymbol{\Sigma}(\rho_1) = \rho_1\mathbf{E}\mathbf{E}^T + (1-\rho_1) \mathbf{G}\mathbf{G}^T$.\\

Note that $\boldsymbol{\Sigma}$ only depends on $\rho_1$. 
To efficiently implement this, we perform a grid search over $\rho_1$ with as little as 10 possible values, thus only slowing down the original method by a factor of 10.

\section{Statistical testing}

To test for GxE interactions we apply the same strategy that is used in the StructLMM method, which in turn adopts the strategy described in Lippert et al \cite{lippert2011fast}.
% Rao et al \cite{rao1948large}.

In this section I describe the key steps.
First, we define score-based test statistic Q as:

\begin{equation}
    Q = \frac{1}{2}\mathbf{y}^T\mathbf{P}_0 \frac{\partial \mathbf{K}}{\partial \theta}\mathbf{P}_0 \mathbf{y} 
\end{equation}

Where $\mathbf{K}$ is the full covariance matrix (\eqref{eq:scStructLMM}):

\begin{equation}\label{eq:full_K_scStructLMM}
    \mathbf{K} = \sigma_{GxE}^2diag(\mathbf{g})\mathbf{E}\mathbf{E}^Tdiag(\mathbf{g}) +  \sigma_e^2 \mathbf{E}\mathbf{E}^T + \sigma_g^2 \mathbf{G}\mathbf{G}^T+ \sigma_n^2 \mathbf{I}
\end{equation}

and 

\begin{equation}
    \mathbf{P}_0 = \mathbf{K}_0^{-1}-\mathbf{K}_0^{-1}\mathbf{X}(\mathbf{X}^T\mathbf{K}_0^{-1}\mathbf{X})^{-1}\mathbf{X}^T\mathbf{K}_0^{-1}
\end{equation}

is a matrix that projects out the fixed effects \cite{lippert2011fast, lippert2014greater}.

In our case (\eqref{eq:scStructLMM}), the fixed effects include covariates and the persistent effect of the variant tested: $\mathbf{X} = [\mathbf{W}, \mathbf{g}]$\\

Using \eqref{eq:full_K_scStructLMM} and considering the parameter $\theta = \sigma_{GxE}^2$

\begin{equation}
    \frac{\partial \mathbf{K}}{\partial \sigma_{GxE}^2} = diag(\mathbf{g})\mathbf{E}\mathbf{E}^Tdiag(\mathbf{g})
\end{equation}

Let us define $\mathbf{K}_1 = diag(\mathbf{g})\mathbf{E}\mathbf{E}^Tdiag(\mathbf{g})$:

\begin{equation}
    Q = \frac{1}{2}\mathbf{y}^T\mathbf{P}_0 \mathbf{K}_1\mathbf{P}_0 \mathbf{y} 
\end{equation}

As in \eqref{eq:StructLMM-int_H1}, $H_1: \sigma_{GxE}^2>0$, noting that as a variance parameter, $\sigma_{GxE}^2$ can only take positive values.
As a result, the Score test statistic $Q$ does not follow the usual $\chi^2_i$ distribution (see \eqref{eq:lagrange_multiplier_univariate}), but instead a mixture of  $\chi^2$ distributions.
I refer the donor specifically at the supplementary methods from \cite{lippert2014greater} for a proof.

\begin{equation}
    Q \sim \sum_i \lambda_i \chi^2_1 
\end{equation}

Where $\lambda_i$'s are the non-zero eigenvalues of $\frac{1}{2}\mathbf{P}_0^{\frac{T}{2}} \frac{\partial\mathbf{K}}{\partial \theta} \mathbf{P}_0^{\frac{1}{2}}$.\\

Now since the eigenvalues of $AA^T$ are the same as those of $A^TA$, we can re-arrange and compute $\lambda_i$'s as the eigenvalues of:

\begin{equation}
    \frac{1}{2}\frac{\partial\mathbf{K}}{\partial \theta}^{\frac{T}{2}} \mathbf{P}_0 \frac{\partial\mathbf{K}}{\partial \theta}^{\frac{1}{2}}
\end{equation}

instead.

To evaluate the significance of the score-best test statistic $Q$ we use the approach described in in SKAT \cite{wu2011rare}, thereby using the Davies exact method \cite{davies1980algorithm} to compute the corresponding p values, and switching to the modified moment matching approximation method (Lee et al 2012, Duchesne and De Micheaux 2010, Liu et al 2009 \cite{liu2009new, lee2012optimal}) when this fails to converge.


Various methods have been proposed to compute the tail probability of the mixture of 1-dof $\chi^2$ distributions. 
For example, we can use the Davies method (Davies, 1980)\cite{davies1980algorithm}, the moment‐matching‐based noncentral $\chi^2$ approximation method \cite{liu2009new, lee2012optimal}(Liu et al., 2009; Lee et al., 2012b), or the saddlepoint approximation method (Kuonen, 1999). 
The Davies methos is the most accurate but can be be computationally expensive.
The moment-matching approximation is anticonservative and could lead to inflated type I errors especially for small significance levels.
SKAT: Davies + Liu
SKATh: Davies + saddle
\cite{wu2016efficient}

% \section{Definition of covariance matrix}

% E?

% PCs, MOFA, HVGs?
% transformation?
% normalization/scaling


% \section{Simulation data}

% First, we applied the model on simulated data.

% \subsection{Simulation strategy}
% Briefly, from the full model (eq. 6.1) we simulate only an intercept as covariate:  

% \begin{equation}
%  \mathbf{y} = \mathbf{y}_0 + \mathbf{g}\beta_G + \mathbf{g} \odot \boldsymbol{\beta_{GxE}} + \mathbf{e} + \mathbf{u} + \boldsymbol{\psi} 
% \end{equation}

% We column normalize all terms so that the total variance sums to 1.
% We set the variance explained by both genetic terms $var_G+var_{GxE}=\sigma_0^2$, and call the rest $v = 1-\sigma_0^2$.
% Further, we regulate the amount of variance driven by GxE using an additional weighting factor $\rho_0$, such that: $var_G = (1-\rho_0)\sigma_0^2$ and $var_{GxE} = \rho_0\sigma_0^2$
% For simplicity, we set the variances explained by the last three terms to be the same:
% $\sigma_E^2 = \sigma_g^2 = \sigma_n^2 = v/3$

% \subsection{Calibration analysis}

% Using simulated data, we checked that our model was calibrated both in the case of no genetic effects at all (i.e. $\sigma_0^2 = 0$) and in the case of persistent effects only (no GxE, i.e. $\rho_0 = 0$).

% \subsection{Comparison with Struct LMM v0}

% Next, we compared our model to the original StructLMM model.
% StructLMM is expected to have issues in the presence of extended repeated structure, which it is not equipped to deal with.
% To address this, we simulated various numbers of repeats per donor - mimicking cells -  ranging from 10 to 500.
% Indeed, we observe over-inflation of StructLMM in the presence of many repeats making the model not calibrated (Fig. Xa).
% sc-StructLMM, on the other hand, is nicely calibrated (Fig. Xb).

% \subsection{Comparison with standard interaction test}

% We then performed power analysis when comparing our model with one where the environments are modelled as fixed effects (see Section 2.4.2).

\newpage

% \section{Real data}
\section{Application to differentiating iPS cells}

We applied the model on the data described in Chapter 4.
% We applied our model on a recently published dataset \cite{cuomo2020single}. 
Here, cells from 125 donors are differentiated from a pluripotent state, to definitive endoderm. 
We use 
principal components (PCs) 
% 10 MOFA factors estimated from the data \cite{argelaguet2018multi}
% 500 independent ($r^2<0.2$) highly variable genes (HVGs)
to capture various aspects of the variation in gene expression in the dataset, which represent cell states and types. 
In particular, PC1 nicely aligns with the differentiation axis. 
We observe an increase in our power to identify interaction eGenes (genes with at least one interaction eQTL) as we increase the number of PCs used as environments, with then a plateau at 50 PCs (Fig.X).
Reassuringly, we can recapitulate most (XX\%) of the dynamic eQTL described in the original paper, which were detected using solely PC1, and a fixed effect linear mixed model (Methods). 
However, we show that we have increased power using our model. 
Furthermore, the overlap with our interaction eQTL and the dynamic eQTL identified in Cuomo \textit{et al}. decreases, the more PCs we include (Fig. XX).\\



% As environments, we used
% the first 10 principal components


% PCs, MOFA, HVGs?
% transformation?
% normalization/scaling




% \section{Upstream analysis}

% One of the key steps in running this method is choosing the environmental factors.
% We expect most applications to be for scRNA-seq datasets only (and genotypes).
% This means that the environments need to be estimated from the transcriptomic data, which is also used as phenotype, causing concerns of circularity.


% A user might have 
% PCA
% MOFA (single omic)

% \section{Downstream analysis} 

% \subsection{Estimate cell-specific genetic effects}

\section{Predicting cell-specific effect sizes driven by GxE}

Using our model we are also able to predict a continuous genetic effect.
 

\begin{equation}
    \mathbf{f}(\mathbf{X}) \sim \mathrm{GP}(\mathbf{m}(\mathbf{x}), k(\mathbf{x},\mathbf{x}^T))
\end{equation}

Out of sample prediction for $\mathbf{f}_*$'s best estimator is its BLUP, defined as its expected value condition on $\mathbf{f}$ and $\mathbf{X},\mathbf{X}_*$:

\begin{equation}
    E[\mathbf{f}_*|\mathbf{f}] = \mathbf{m}_* +k(\mathbf{X}_*,\mathbf{X})k(\mathbf{X},\mathbf{X})^{-1}(\mathbf{f}-\mathbf{m})
\end{equation}

In our case,

\begin{equation}
    \mathbf{f}(\mathbf{X}) = \mathbf{y} = \mathbf{W}\boldsymbol{\alpha}+\mathbf{g}\beta_G+\mathbf{g}\boldsymbol{\beta}_{GxE}+\mathbf{e} + \mathbf{u} + \boldsymbol{\psi}
\end{equation}

with ($\mathbf{X} = {\mathbf{W},\mathbf{g},\mathbf{E},\mathbf{K}}$):

\begin{equation}
    \mathbf{m}(\mathbf{X}) = \mathbf{W}_*\boldsymbol{\alpha}_{*}+\mathbf{g}_*\beta_G
\end{equation}

\begin{equation}
    k(\mathbf{X},\mathbf{X}) = \sigma_{GxE}^2(\mathbf{g}\odot\mathbf{E})(\mathbf{g}\odot\mathbf{E})^T
\end{equation}

\begin{equation}
\mathbf{y}_{*}^{BLUP} = E[\mathbf{y}_*|\mathbf{y}] = 
\mathbf{W}_*\boldsymbol{\alpha}_{*}+\mathbf{g}_*\beta_G+\mathbf{E}_*\gamma + 
k(\mathbf{X}_*\mathbf{X})K^{-1}(\mathbf{y}-\mathbf{W}\boldsymbol{\alpha}-\mathbf{g}\beta_G-\mathbf{E}\gamma)
\end{equation}



\section{Discussion}

upstream analysis
One of the key steps in running this method is choosing the environmental factors.
We expect most applications to be for scRNA-seq datasets only (and genotypes).
This means that the environments need to be estimated from the transcriptomic data, which is also used as phenotype, causing concerns of circularity.\\

downstream analysis
which environments are responsible\\

\begin{itemize}
    \item GLMM (Poisson, NB)
    \item additional covariances?
    \item other applications (longitudinal data)
\end{itemize}
